\begin{center}
	\pagenumbering{gobble}

	\begin{figure}[h!]
        \begin{subfigure}[h!]{0.49\linewidth}
		    \includegraphics[scale=0.3,left]{images/ox_brand_cmyk_pos.png}
        \end{subfigure}
        \begin{subfigure}[h!]{0.5\linewidth}
            \includegraphics[scale=0.3,right]{images/snoplus_logo_big.pdf}
        \end{subfigure}
	\end{figure}

	\vfill

	\hrule
	\vspace{1cm}
	\Huge {Studies of Optical Scattering and First Measurement of the Solar Neutrino Oscillation Parameters in the SNO+ detector}\\
	\vspace{1cm}
	\hrule

	\vfill


	\Large Daniel Cookman\\
	\large Confirmation of Status Report\\
    \large Hilary Term, 2022\\
	\large Department of Physics\\
    \large University of Oxford\\

	\vfill

\end{center}

	\normalsize\textbf{Abstract}\\
	This report contains four main sections. A draft of the introduction to be included within my thesis is given. Following this, the SNO+ detector is briefly overviewed, as well as its scattering calibration system SMELLIE. The main body of the report contains work performed to improve the simulation quality of this calibration system: uncertainties in the angular emission distribution have been improved significantly through the use of more calibration data as well as a careful maximum likelihood estimate statistical approach. In addition, a new generator for simulating these SMELLIE events has been built, which dramatically reduced simulation times by three orders of magnitude. Finally, an outline of my thesis is included, describing the planned contents for each chapter, along with what work is left to be completed in each. A description of what data will be used as part of this thesis is included.
