\section{Thesis Introduction}\label{sect:Introduction}

Ever since their postulation by Wolfgang Pauli almost 100 years ago, the observational consequences of neutrinos have forced us to understand our Universe better. The apparent violation of energy and angular momentum conservation of radioactive $\beta$-decay led Pauli to propose the particle in the first place~\cite{noauthor_rapports_1935}, but the same pattern of surprising results has continued apace. Pauli suspected that the neutrino could never be detected, and then Cowan \& Reines were able to detect (anti)neutrinos from a nuclear reactor before Pauli's death~\cite{reines_neutrino_1956,cowan_detection_1956}. In the subsequent decades, two further flavours of neutrino were detected (muon and tau neutrinos)~\cite{danby_observation_1962,kodama_observation_2001}, the three flavours coming from a variety of sources: not only the aforementioned radioisotopes and nuclear reactors, but also particle accelerators~\cite{danby_observation_1962}, the atmosphere~\cite{abe_atmospheric_2018}, the Earth itself~\cite{agostini_comprehensive_2020}, a distant supernova~\cite{hirata_observation_1987}, and even our Sun~\cite{cleveland_measurement_1998}. To this day, a major puzzle remains: we observe neutrinos oscillating between their flavours when they propagate through space, which mean they must have a mass. But in the Standard Model (SM) of particle physics, there is no way for neutrinos to obtain mass~\cite{particle_data_group_review_2020}!

SNO+ is an experiment currently running whose flagship physics goal is the search for `smoking-gun' evidence of a means by which neutrinos could get their mass: neutrinoless double-beta decay~\cite{albanese_sno_2021}. The detector is also able to perform a wide variety of other neutrino physics, including measurement of solar neutrino oscillations --- it is measuring these oscillations that forms a major part of this thesis. The neutrino physics community currently has a good phenomenological model for how neutrinos oscillate, known as the PMNS matrix~\cite{particle_data_group_review_2020}. Most of the parameters within this matrix have been measured precisely by multiple experiments. However, it is quite possible that this oscillation model is wrong, and so further precise measurements of its parameters are highly valued. Solar neutrinos allow us to probe two of these parameters, known as $\Delta m^{2}_{21}$ and $\theta_{12}$. SNO+ in particular is uniquely poised to be able to measure these parameters not only by solar neutrinos, but also anti-neutrinos generated from reactors. Combining these results together could give a world-leading limit. My work shown here will not only be the first analysis of data from the detector in pursuit of measuring solar neutrino oscillations, but also will provide a sensitivity study for how we expect this preliminary result to improve with more data in the future.

Of course, for SNO+ to be able to make any measurements with precision requires a well-calibrated detector. In particular, understanding how light interacts with the detector is of paramount importance: it is light that gets generated by the physics events we would like to study within our detector. One of the principal aspects by which that light can interact is by scattering off of the detector medium. When there is a high degree of scattering in the detector, light will bounce around a lot before being detected. In this scenario, our ability to reconstruct the position of the inciting physics event becomes limited, which is critical for precise analysis work. SMELLIE is a calibration system for SNO+ designed to measure and monitor this optical scattering. This forms the other major part of my thesis: taking data with SMELLIE, and then analysing that data to understand the detector's scattering properties. This measurement has never before been done in a large liquid scintillator experiment, such as SNO+.

I will start the thesis in Chapter 1 by giving a run-down of the theoretical concepts critical to understanding the neutrino physics being researched here. Following this, Chapter 2 is a practical look at how the SNO+ detector works: going from how physics events produce light in the detector to how those photons get detected, and then subsequently packaged into data that gets stored and analysed. It will also cover some of the basic in how this data gets calibrated and reconstructed for high-level analyses.

Chapter 3 will cover the theory necessary to understanding optical scattering in a fluid, and how we model this scattering in our simulations. Then, Chapter 4 takes a look at the specifics of the SMELLIE calibration system: its hardware, and how it gets commissioned for effective use. After this look at hardware, we jump to the software side of things and consider the simulation of SMELLIE events in Chapter 5. A number of major improvements were made here, notably making the simulation happen orders of magnitude faster, and also building a method to calibrate this simulation with far more data than used before.

Chapter 6 is split into two parts, both covering how SMELLIE can be used to calibrate the detector. In the first section, a careful comparison of data between when the detector was filled with water versus scintillator allows us to measure the extinction length of the scintillator as a function of wavelength, and monitor this over time. The second section then covers how a more extensive analysis of SMELLIE data, and comparing to simulations can lead to one being able to make a measurement of scattering within the detector in both the water and scintillator phases.

Chapter 7 is also split into two parts covering my analysis of solar neutrino oscillations. I first build up the analysis approach with simulated data, and perform a sensitivity study of the detector to measuring the solar oscillation parameters. The second section then applies this to data taken during the scintillator phase of the experiment, and the first ever parameter estimates are obtained for SNO+. Finally we summarise all of the work described in this thesis in Chapter 8, the conclusion.